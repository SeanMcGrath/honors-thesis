%!TEX root = ../dissertation.tex

\chapter{Discussion}

\newthought{The preceding results} point clearly to the relative efficacy of our new temperature-dependent cellulose simulation protocol. Figure \ref{fig:single_gdp} shows the inadequacy of conventional techniques when applied to a single cellobiose unit; the computed spectrum, shown in black, fails to agree with the experimental spectrum in red. The simulation places the main (\SI{1100}{cm^{-1}}) peak at the wrong frequency, and predicts a large peak at \SI{1400}{cm^{-1}} that is not observed. Generally, the simulated spectrum is much sharper with more distinct peaks than the experimental one.

The application of the new protocol produces a simulated spectrum much closer to that of real cellulose, as shown in Figure \ref{fig:money_single}. Here, the main peak is located properly, and has the correct shape, with two distinct shoulders to the right of the main peak. The spectrum appears much smoother, and the peaks are generally confined to low (0\textendash600 \si{cm^{-1}}), middle (1000\textendash1200 \si{cm^{-1}}), and high (1300\textendash1500 \si{cm^{-1}}) frequency regions, as is seen in the cellulose spectrum. There are some notable deviations between the two data sets, particularly a distinct peak at approximately \SI{1500}{cm^{-1}} that appears in the simulation but not in the experiment. We suspect such deviations arise from the fact that our isolated cellobiose units do not capture the effects of interchain hydrogen bonding that occurs in real cellulose, and presumably suppresses certain vibrational modes.

The peak assignments listed in Table \ref{tbl:assignments} suggest that the three frequency regions discussed above map clearly onto three distinct types of vibrations. The low frequency region is comprised of large-scale vibrations distributed over many atoms and bonds in the glucose rings. The middle frequencies about the main peak are caused by heavy atom stretching, or changes in the bond lengths between carbon and oxygen atoms. Finally, the high frequency region corresponds to bending modes, where bond angles involving both heavy atoms and hydrogen atoms change. There is also another vibrational region, consisting of hydrogen-oxygen and hydrogen-carbon stretching modes, that occurs above \SI{3200}{cm^{-1}}; however, due to a lack of comparable experimental data in this region, it will be neglected from further analysis until such data becomes available. It is worth noting that these rough assignments agree with the conclusions reached in past work on cellulose's vibrational spectra\cite{RefWorks:39}.

Figures \ref{fig:main_peak_shift} and \ref{fig:1330_peak_shift} show that the simulation is capable of accurately predicting the effect of changing temperature on mode frequency in the stretching region, but not in the bending region; for the main stretching peak, the simulation predicts that the frequency of the peak should shift by \SI{-0.02}{cm^{-1}} per Kelvin, in excellent agreement with an observed redshift of \SI{-0.03}{cm^{-1}} per Kelvin. We attribute this redshift to the anharmonicity of the Lennard-Jones potential describing atom-to-atom repulsion and attraction; essentially, at higher temperatures, the atoms in the molecule vibrate more vigorously, and go both closer and farther away from each other as stretching vibrations occur. For a perfectly harmonic potential, this would not affect the frequency, but for a Lennard-Jones potential, larger separation distances lead to a restoring force that is weaker than that of a simple quadratic potential. Therefore atoms that travel farther from the equilibrium distance spend more time in the relatively weak anharmonic region, feeling a weaker average restoring force and vibrating at a lower frequency.

The inability of the simulations to capture the behavior of the bending modes is attributed to the fact that the majority of the bending motions describe the behavior of hydrogen atoms attached to in-ring carbons. These are the precisely the hydrogens that had to be added manually to the otherwise temperature-dependent cellobiose structures. Therefore these hydrogens are essentially unaffected by the simulated temperature, and their bending motions do not change as the simulation temperature increases. This is the cause of the slope mismatch evident in Figure \ref{fig:1330_peak_shift}. It is a distinct shortcoming of this computational approach that it does not permit quantitative analysis of modes in the bending region.

The data in Table \ref{tbl:temp_freqs} represent our most comprehensive attempt to understand the effect of increasing temperature on different types of vibrations. It lists the computed frequency of all 129 vibrational modes of cellobiose at each examined temperature up to 500K. For each mode, a linear regression has been performed, and the slope of the best-fit line describing the linear relationship between temperature and frequency is listed. This is accompanied by the absolute value of the Pearson product-moment correlation function, which measures the linearity of the temperature-frequency relationship. A value of 1 for this function implies a perfect linear relationship between the two variables, and a value of 0 implies no linear relationship.

The results indicate that the vibrations most clearly affected by increasing temperature are the stretching vibrations. No non-stretching mode has a linear correlation function value higher than 0.43; carbon-oxygen and hydrogen-oxygen stretches have correlations as high as 0.9.	Low-frequency skeletal modes display little linear dependence on temperature, with correlations below 0.25 and typically below 0.1. The bending modes also have very weak linear correlations, though for the reasons described above, simulation data for these modes are probably not useful.

Vibrations associated with movement of the glycosidic linkage are affected by temperature. Modes 50 and 55, asymmetric and symmetric stretches of the linkage respectively, both display a redshift with respective linear correlations of 0.78 and 0.80 respectively.
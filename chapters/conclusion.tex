%!TEX root = ../dissertation.tex
\chapter{Conclusion}

\newthought{The results discussed} in the preceding section suggest that the work presented here has brought us to two main conclusions. The first is that we now have a robust technique for predicting and calculating the Raman spectrum of complex polymers at elevated temperatures. The polymer investigated here was cellulose, but we see no reason that similar polymers could not be treated in the same way. This finding is significant because, to our knowledge, no technique for such temperature-dependent spectral analysis currently exists in the chemical literature.

The second finding is the emergence of two broad peaks in the experimental Raman spectrum of cellulose at 483K that are not predicted by our non-reactive model. These peaks indicate not just that the structural/vibrational behavior of cellulose begins to change at that temperature, they may potentially indicate exactly how that change is occurring. Identifying the nature of those peaks is the primary goal as this research continues.

Though we cannot yet confidently say what is causing new peak emergence, we have produced and are in the midst of expanding on a powerful data set that attempts to quantify the effects of temperature on the different kinds of vibrational modes. Our first attempt at this analysis is seen in Table \ref{tbl:temp_freqs}. We note that this particular type of data is not attainable with any other method known to date; it is common for experimental spectroscopists to analyze the shifting of observed peaks, but in this way it is impossible to identify the exact underlying vibration that is being shifted. Our simulation makes such conclusions possible, and thus represents a significant increase in the analytic power available to physical chemists.

We regret that, due to time constraints on the production of this thesis, this further analysis is incomplete. However, we do believe that it will prove to provide powerful new insights into cellulose thermochemistry. Some key areas for new work include tabulating the Raman intensity for each vibrational mode at each temperature, and carrying out peak assignments at every simulated temperature. In this way we can quantify the possible effects of temperature on spectral intensity and the nature of the vibrations themselves, in addition to simply the vibrational frequency.

All of this work, of course, is in service to an overarching goal of improving the efficiency of pyrolytic biofuel production. Understanding changes in the Raman spectrum of cellulose is simply another step towards more efficient energy production and a more sustainable society.
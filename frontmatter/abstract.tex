Liquid fuels and plastics are generally made from environmentally unfriendly and nonrenewable sources such as petroleum. These fossil fuels are a major source of carbon dioxide emissions, and their use is the likely cause of phenomena such as acid rain and anthropogenic global warming. Biofuels, made from plants that remove carbon dioxide from the air during their growth, have the potential to replace fossil fuels as a sustainable and carbon-neutral energy source. Attempts at large-scale biofuel production have been made, primarily focusing on bioethanol made from corn and other grains. However, researchers believe that biofuel made from cellulose has the potential to be more efficient and effective. 


Cellulose is a polymer that is the main component of plant matter, and it can be broken down by heat into a variety of precursor chemicals that serve as feedstocks for clean fuel synthesis. However, this pyrolytic process is currently poorly understood and hard to control precisely. In this study, we advance a new method for modeling the vibrational behavior and spectra of cellulose, in hopes of gaining a better understanding of the effects of temperature on this chemical. This method employs classical models of cellulose to simulate changes in its structure as it heats, and quantum techniques to calculate the Raman spectra associated with these high-temperature geometries. The data obtained are compared with experimentally observed spectra at the same temperatures and searched for evidence of early steps in the pyrolysis process as well as revised assignments for peaks in the vibrational spectra. Simulation data reveals both high and low-frequency deviations from experiment, indicating non-classical reactive behavior of cellulose at temperatures below the chemical's melting point. This work informs possible routes for improving cellulose pyrolysis and probes the limits of a classical model of cellulose heating.